%lab report 2 of bresenhaum's line drawing algorithm and midpoint circle drawing algorithm in graphics.h
\documentclass[12pt]{article}

\usepackage{amsmath}
\usepackage{graphicx}

\begin{document}
	
	\title{Bresenham's Line Drawing Algorithm And Midpoint Circle Drawing Algorithm}
	\author{Jenish Pant}
	\date{\today}
	\maketitle

	\section{Objective}
	To implement Bresenham's Line Drawing Algorithm and Midpoint Circle Drawing Algorithm in graphics.h
	\section{Theory}
	\subsection{Bresenham's Line Drawing Algorithm}
	Bresenham's line algorithm is an algorithm that determines the points of an n-dimensional raster that should be selected in order to form a close approximation to a straight line between two points. It is commonly used to draw lines on a computer screen, as it uses only integer addition, subtraction and bit shifting, all of which are very cheap operations in standard computer architectures. It is an incremental error algorithm. It is one of the earliest algorithms developed in the field of computer graphics.\\
	\subsection{Midpoint Circle Drawing Algorithm}
	The midpoint circle algorithm is an algorithm used to determine the points needed for rasterizing a circle. Bresenham's circle algorithm is derived from the midpoint circle algorithm.\\

	\section{Algorithm}
	\subsection{Bresenham's Line Drawing Algorithm}
	\begin{enumerate}
		\item Input the two end points of the line, storing the left end point in (x0, y0).
		\item Plot the point (x0, y0).
		\item Calculate constants dx, dy, 2dy, and 2dy - 2dx, and obtain the starting value for the decision parameter as p0 = 2dy - dx.
		\item At each xk along the line, starting at k = 0, perform the following test:
		\begin{enumerate}
			\item If pk < 0, the next point to plot is (xk+1, yk) and pk+1 = pk + 2dy.
			\item If pk >= 0, the next point to plot is (xk+1, yk+1) and pk+1 = pk + 2dy - 2dx.
		\end{enumerate}
		\item Repeat step 4 dx times.
	\end{enumerate}
	\subsection{Midpoint Circle Drawing Algorithm}
	\begin{enumerate}
		\item Input the radius r and circle center (xc, yc), and obtain the first point on the circumference of a circle centered on the origin as (0, r).
		\item Calculate the initial value of the decision parameter as p0 = 5/4 - r.
		\item At each xk along the circumference, starting at k = 0, perform the following test:
		\begin{enumerate}
			\item If pk < 0, the next point to plot is (xk+1, yk) and pk+1 = pk + 2xk+1 + 1.
			\item If pk >= 0, the next point to plot is (xk+1, yk-1) and pk+1 = pk + 2xk+1 - 2yk+1 + 1.
		\end{enumerate}
		\item Repeat step 3 until x >= y.
	\end{enumerate}
	\section{Source Code}
	\subsection{Bresenham's Line Drawing Algorithm}
	\begin{verbatim}
	#include<stdio.h>
	#include<graphics.h>
		
	\end{verbatim}		

	\subsection{Midpoint Circle Drawing Algorithm}
	\begin{verbatim}
	#include<stdio.h>
	#include<graphics.h>

	\end{verbatim}

	\section{Output}

	%\subsection{Bresenham's Line Drawing Algorithm}
	%\begin{figure}[h!]
	%	\centering
	%	\includegraphics[width=0.7\linewidth]{bresenham}
	%	\caption{Bresenham's Line Drawing Algorithm}
	%	\label{fig:bresenham}
	%\end{figure}
	%\subsection{Midpoint Circle Drawing Algorithm}
	%\begin{figure}[h!]
	%	\centering
	%	\includegraphics[width=0.7\linewidth]{circle}
	%	\caption{Midpoint Circle Drawing Algorithm}
	%	\label{fig:circle}
	%\end{figure}

	\section{Conclusion}
	We have implemented Bresenham's Line Drawing Algorithm and Midpoint Circle Drawing Algorithm in graphics.h
\end{document}

